\section{Introduction}

In the realm of environmental conservation and biodiversity research, the development of a machine learning algorithm capable of accurately classifying bird species from images represents a groundbreaking advancement. This technological endeavor addresses a critical challenge with far-reaching implications in various fields. 

The cornerstone of this project is its potential impact on conservation and biodiversity monitoring. Birds are often considered key indicators of an ecosystem's health. Therefore, the ability to automatically identify bird species is an invaluable asset for conservationists. This capability is particularly crucial in conservation areas, where monitoring the presence and diversity of bird species can yield vital information about the state of biodiversity and the effectiveness of ongoing conservation efforts.

Another significant aspect of this project is the efficiency and scalability it offers. Traditionally, the identification of bird species has been reliant on expert knowledge, a process that is both time-intensive and limited in scope. An automated system, on the other hand, can rapidly process a large volume of images, thereby facilitating the monitoring of more extensive areas and a greater variety of species than what is currently possible through manual methods.

Beyond its scientific and conservation applications, this tool also holds promise for public engagement and education. By integrating an automated bird identification system into apps and platforms used by bird watchers and nature enthusiasts, it can enhance their experience and knowledge. This, in turn, helps in raising public awareness about various bird species and broader conservation issues.

The input to our algorithm is an image. We then use a neural network to output a predicted bird species.